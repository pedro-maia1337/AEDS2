%%%%%%%%%%%%%%%%%%%%%%%%%%%%%%%%%%%%%%%%%%%%%%%%%%%%%%%%%%%%%%%%%%%%%%%%%%%%%%%%%%%%%%%%%%%%%%%%%%%%%%%
%%%%%%%%%%%%%% Template de Artigo Adaptado para Trabalho de Diplomação do ICEI %%%%%%%%%%%%%%%%%%%%%%%%
%% codificação UTF-8 - Abntex - Latex -  							     %%
%% Autor:    Fábio Leandro Rodrigues Cordeiro  (fabioleandro@pucminas.br)                            %% 
%% Co-autores: Prof. João Paulo Domingos Silva, Harison da Silva e Anderson Carvalho		     %%
%% Revisores normas NBR (Padrão PUC Minas): Helenice Rego Cunha e Prof. Theldo Cruz                  %%
%% Versão: 1.1     18 de dezembro 2015                                                               %%
%%%%%%%%%%%%%%%%%%%%%%%%%%%%%%%%%%%%%%%%%%%%%%%%%%%%%%%%%%%%%%%%%%%%%%%%%%%%%%%%%%%%%%%%%%%%%%%%%%%%%%%

\section{\esp Código da Aplicação}

\begin{adjustwidth}{-1cm}{-1cm} % Reduz as margens laterais só para este bloco
\begin{lstlisting}[caption={Código da aplicação}, label={lst:codigo_quicksort}]
import java.util.concurrent.ThreadLocalRandom;

class Quicksort extends Geracao {

    public Quicksort(){ super(); }
    public Quicksort(int tamanho){ super(tamanho); }

    @Override
    public void sort(int i) {
        if(i==1) QuickSortFirstPivot(0,n-1);
        if(i==2) QuickSortLastPivot(0,n-1);
        if(i==3) QuickSortRandomPivot(0,n-1);
        if(i==4) QuickSortMedianOfThree(0,n-1);
    }

    private void QuickSortFirstPivot(int esq, int dir) {
        int i=esq, j=dir, pivo=array[i];
        while(i<=j){
            while(array[i]<pivo){ cmp++; i++; }
            while(array[j]>pivo){ cmp++; j--; }
            if(i<=j){ swap(i,j); mov+=3; i++; j--; }
        }
        if(esq<j) QuickSortFirstPivot(esq,j);
        if(i<dir) QuickSortFirstPivot(i,dir);
    }

    private void QuickSortLastPivot(int esq, int dir) {
        int i=esq, j=dir, pivo=array[dir];
        while(i<=j){
            while(array[i]<pivo){ cmp++; i++; }
            while(array[j]>pivo){ cmp++; j--; }
            if(i<=j){ swap(i,j); mov+=3; i++; j--; }
        }
        if(esq<j) QuickSortLastPivot(esq,j);
        if(i<dir) QuickSortLastPivot(i,dir);
    }

    private void QuickSortRandomPivot(int esq, int dir) {
        int randomPivot=ThreadLocalRandom.current().nextInt(esq,dir+1);
        int i=esq, j=dir, pivo=array[randomPivot];
        while(i<=j){
            while(array[i]<pivo){ cmp++; i++; }
            while(array[j]>pivo){ cmp++; j--; }
            if(i<=j){ swap(i,j); mov+=3; i++; j--; }
        }
        if(esq<j) QuickSortRandomPivot(esq,j);
        if(i<dir) QuickSortRandomPivot(i,dir);
    }

    private void QuickSortMedianOfThree(int esq, int dir) {
        int i=esq, j=dir, pivo=array[(dir+esq)/2];
        while(i<=j){
            while(array[i]<pivo){ cmp++; i++; }
            while(array[j]>pivo){ cmp++; j--; }
            if(i<=j){ swap(i,j); mov+=3; i++; j--; }
        }
        if(esq<j) QuickSortMedianOfThree(esq,j);
        if(i<dir) QuickSortMedianOfThree(i,dir);
    }
}
\end{lstlisting}
\end{adjustwidth}


\section{\esp Pivô como Primeiro Elemento}

% ==============================
% Tabela - Pivô Primeiro Elemento
% ==============================
\begin{table}[H]
    \centering
    \scriptsize
    \caption{Comparação dos cenários com pivô sendo o primeiro elemento, considerando $n=1000$.}
    \label{tab:pivo_primeiro}
    
    \resizebox{\textwidth}{!}{%
    \begin{tabular}{|l|c|c|c|c|c|}
        \hline
        \textbf{Cenário} & \textbf{Ordem de Complexidade} & \textbf{Fórmula Comparações} & 
        \textbf{Fórmula Movimentações} & \textbf{Nº Comparações} & \textbf{Nº Movimentações} \\ \hline
        
        \textbf{Array Crescente} 
        & $O(n^2)$ 
        & $C(n)=\frac{n(n-1)}{2}$ 
        & $M(n)=3\cdot(n-1)$ 
        & $499{,}500$ 
        & $2{,}997$ \\ \hline
        
        \textbf{Parcialmente Ordenado} 
        & Entre $O(n \log n)$ e $O(n^2)$ 
        & *** 
        & *** 
        & $8{,}040$ 
        & $20{,}751$ \\ \hline
        
        \textbf{Array Aleatório} 
        & $O(n \log n)$ 
        & $C(n)=n\cdot\log_2 n$ 
        & $M(n)=3\cdot n\cdot\log_2 n$ 
        & $7{,}902$ 
        & $9{,}190$ \\ \hline
    \end{tabular}
    }
\end{table}

\begin{figure}[H]
    \centering
    \caption{Execução do Array Crescente}
    \includegraphics[width=1.0\textwidth]{figuras/pivo_primeiro/Array_crescente.png}
    \label{fig:pivo_primeiro_crescente}
\end{figure}

\begin{figure}[H]
    \centering
    \caption{Execução do Array Parcialmente Ordenado}
    \includegraphics[width=1.0\textwidth]{figuras/pivo_primeiro/Array_parcialmente_ordenado.png}
    \label{fig:pivo_primeiro_parcialmente}
\end{figure}

\begin{figure}[H]
    \centering
    \caption{Execução do Array Aleatório}
    \includegraphics[width=1.0\textwidth]{figuras/pivo_primeiro/Array_random.png}
    \label{fig:pivo_primeiro_random}
\end{figure}

\begin{figure}[H]
    \centering
    \caption{Gráfico geral do cenário com pivô primeiro elemento}
    \includegraphics[width=1.0\textwidth]{figuras/pivo_primeiro/pivo_primeiro.png}
    \label{fig:pivo_primeiro_grafico}
\end{figure}

% ==============================
% Seção: Pivô Último Elemento
% ==============================
\section{\esp Pivô como Último Elemento}

\begin{table}[H]
    \centering
    \scriptsize
    \caption{Comparação dos cenários com pivô sendo o ultimo elemento, considerando $n=1000$.}
    \label{tab:pivo_primeiro}
    
    \resizebox{\textwidth}{!}{%
    \begin{tabular}{|l|c|c|c|c|c|}
        \hline
        \textbf{Cenário} & \textbf{Ordem de Complexidade} & \textbf{Fórmula Comparações} & 
        \textbf{Fórmula Movimentações} & \textbf{Nº Comparações} & \textbf{Nº Movimentações} \\ \hline
        
        \textbf{Array Crescente} 
        & $O(n^2)$ (Pior Caso)
        & $C(n)=\frac{n(n-1)}{2}$ 
        & $M(n)=3\cdot(n-1)$ 
        & $499{,}500$ 
        & $2{,}997$ \\ \hline
        
        \textbf{Parcialmente Ordenado} 
        & Entre $O(n \log n)$ e $O(n^2)$ 
        & *** 
        & *** 
        & $8{,}040$ 
        & $20{,}751$ \\ \hline
        
        \textbf{Array Aleatório} 
        & $O(n \log n)$ (Caso Médio)
        & ***
        & ***
        & $7{,}902$ 
        & $9{,}190$ \\ \hline
    \end{tabular}
    }
\end{table}

\begin{figure}[H]
    \centering
    \caption{Execução do Array Crescente}
    \includegraphics[width=1.0\textwidth]{figuras/pivo_ultimo/array_crescente.png}
    \label{fig:pivo_ultimo_crescente}
\end{figure}

\begin{figure}[H]
    \centering
    \caption{Execução do Array Parcialmente Ordenado}
    \includegraphics[width=1.0\textwidth]{figuras/pivo_ultimo/array_parcialmente_ordenado.png}
    \label{fig:pivo_ultimo_parcialmente}
\end{figure}

\begin{figure}[H]
    \centering
    \caption{Execução do Array Aleatório}
    \includegraphics[width=1.0\textwidth]{figuras/pivo_ultimo/array_random.png}
    \label{fig:pivo_ultimo_random}
\end{figure}

\begin{figure}[H]
    \centering
    \caption{Gráfico geral do cenário com pivô último elemento}
    \includegraphics[width=1.0\textwidth]{figuras/pivo_ultimo/pivo_ultimo.png}
    \label{fig:pivo_ultimo_grafico}
\end{figure}

% ==============================
% Seção: Pivô Aleatório
% ==============================
\section{\esp Pivô Aleatório}

\begin{table}[H]
    \centering
    \scriptsize
    \caption{Comparação dos cenários com pivô aleatório, considerando $n=1000$.}
    \label{tab:pivo_ultimo}
    
    \resizebox{\textwidth}{!}{%
    \begin{tabular}{|l|c|c|c|c|c|}
        \hline
        \textbf{Cenário} & \textbf{Ordem de Complexidade} & \textbf{Fórmula Comparações} & 
        \textbf{Fórmula Movimentações} & \textbf{Nº Comparações} & \textbf{Nº Movimentações} \\ \hline
        
        \textbf{Array Crescente} 
        & $O(n \log n)$ (Caso médio)
        & $C(n)=n\cdot\log_2 n$ 
        & $M(n)=3\cdot n\cdot\log_2 n$ 
        & $9{,}966$ 
        & $29{,}897$ \\ \hline
        
        \textbf{Parcialmente Ordenado} 
        & $O(n \log n)$ (Caso médio)
        & $C(n)=n\cdot\log_2 n$ 
        & $M(n)=3\cdot n\cdot\log_2 n$ 
        & $9{,}966$ 
        & $29{,}897$ \\ \hline
        
        \textbf{Array Aleatório} 
        & $O(n \log n)$ (Caso médio)
        & $C(n)=n\cdot\log_2 n$ 
        & $M(n)=3\cdot n\cdot\log_2 n$ 
        & $9{,}966$ 
        & $29{,}897$ \\ \hline
    \end{tabular}
    }
\end{table}

\begin{figure}[H]
    \centering
    \caption{Execução do Array Crescente}
    \includegraphics[width=1.0\textwidth]{figuras/pivo_aleatorio/array_crescente.png}
    \label{fig:pivo_ultimo_crescente}
\end{figure}

\begin{figure}[H]
    \centering
    \caption{Execução do Array Parcialmente Ordenado}
    \includegraphics[width=1.0\textwidth]{figuras/pivo_aleatorio/array_parcialmente_ordenado.png}
    \label{fig:pivo_ultimo_parcialmente}
\end{figure}

\begin{figure}[H]
    \centering
    \caption{Execução do Array Aleatório}
    \includegraphics[width=1.0\textwidth]{figuras/pivo_aleatorio/array_random.png}
    \label{fig:pivo_ultimo_random}
\end{figure}

\begin{figure}[H]
    \centering
    \caption{Gráfico geral do cenário com pivô último elemento}
    \includegraphics[width=1.0\textwidth]{figuras/pivo_aleatorio/download.png}
    \label{fig:pivo_ultimo_grafico}
\end{figure}

% ==============================
% Seção: Pivô Mediana
% ==============================
\section{\esp Pivô como Mediana dos elementos do array}

\begin{table}[H]
    \centering
    \scriptsize
    \caption{Comparação dos cenários com pivô sendo a mediana, considerando $n=1000$.}
    \label{tab:pivo_ultimo}
    
    \resizebox{\textwidth}{!}{%
    \begin{tabular}{|l|c|c|c|c|c|}
        \hline
        \textbf{Cenário} & \textbf{Ordem de Complexidade} & \textbf{Fórmula Comparações} & 
        \textbf{Fórmula Movimentações} & \textbf{Nº Comparações} & \textbf{Nº Movimentações} \\ \hline
        
        \textbf{Array Crescente} 
        & $O(n \log n)$ (Caso médio)
        & $C(n)=n\cdot\log_2 n$ 
        & $M(n)=3\cdot n\cdot\log_2 n$ 
        & $9{,}966$ 
        & $29{,}897$ \\ \hline
        
        \textbf{Parcialmente Ordenado} 
        & $O(n \log n)$ (Caso médio)
        & $C(n)=n\cdot\log_2 n$ 
        & $M(n)=3\cdot n\cdot\log_2 n$ 
        & $9{,}966$ 
        & $29{,}897$ \\ \hline
        
        \textbf{Array Aleatório} 
        & $O(n \log n)$ (Caso médio)
        & $C(n)=n\cdot\log_2 n$ 
        & $M(n)=3\cdot n\cdot\log_2 n$ 
        & $9{,}966$ 
        & $29{,}897$ \\ \hline
    \end{tabular}
    }
\end{table}

\begin{figure}[H]
    \centering
    \caption{Execução do Array Crescente}
    \includegraphics[width=1.0\textwidth]{figuras/pivo_mediana/array_crescente.png}
    \label{fig:pivo_ultimo_crescente}
\end{figure}

\begin{figure}[H]
    \centering
    \caption{Execução do Array Parcialmente Ordenado}
    \includegraphics[width=1.0\textwidth]{figuras/pivo_mediana/array_parcialmente_ordenado.png}
    \label{fig:pivo_ultimo_parcialmente}
\end{figure}

\begin{figure}[H]
    \centering
    \caption{Execução do Array Aleatório}
    \includegraphics[width=1.0\textwidth]{figuras/pivo_mediana/array_random.png}
    \label{fig:pivo_ultimo_random}
\end{figure}

\begin{figure}[H]
    \centering
    \caption{Gráfico geral do cenário com pivô último elemento}
    \includegraphics[width=1.0\textwidth]{figuras/pivo_mediana/download.png}
    \label{fig:pivo_ultimo_grafico}
\end{figure}

% ==============================
% Conclusão
% ==============================
\section{\esp Considerações e Conclusão}

A classificação lógica da pesquisa em exploratória, descritiva e explicativa representa não apenas uma organização didática do conhecimento metodológico, mas também uma estratégia prática fundamental para a condução científica eficaz. A pesquisa exploratória permite ao pesquisador identificar caminhos e construir bases iniciais sólidas; a pesquisa descritiva oferece um retrato fiel da realidade observada, permitindo comparações e formulações iniciais de padrões; e a pesquisa explicativa propicia o aprofundamento na compreensão dos fenômenos, estabelecendo relações causais que contribuem para o avanço da ciência.

A escolha adequada do tipo de pesquisa a ser realizado deve considerar a maturidade do tema estudado, os objetivos científicos do estudo e as possibilidades metodológicas disponíveis. Mais do que uma escolha técnica, trata-se de uma decisão estratégica que impacta diretamente na qualidade dos dados obtidos e na robustez das conclusões. Assim, conhecer profundamente as características, potencialidades e limitações de cada tipo lógico de pesquisa é condição sine qua non para o desenvolvimento de estudos científicos relevantes, inovadores e socialmente úteis.

\nocite{artigo01}
\nocite{artigo02}
\nocite{artigo03}
\nocite{videoaula}
